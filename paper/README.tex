\documentclass[12pt]{article}
\usepackage[a4paper,margin=1in]{geometry}
\usepackage{hyperref}
\usepackage{setspace}
\usepackage{fancyhdr}
\usepackage{afterpage}
\setstretch{1.15}

\title{\centering \textbf{Project Lambda: Can Artificial Agents Rediscover the Laws of Physics?}}
\author{Henry H. Hicks \\ Tallahassee State College}
\date{2025}

\pagestyle{fancy}
\fancyhf{} % clear header and footer
\renewcommand{\headrulewidth}{0pt}
\renewcommand{\footrulewidth}{0pt}

\begin{document}
\maketitle

\section*{Overview}
\textit{Project Lambda} is an open research initiative exploring whether artificial agents can independently rediscover fundamental physical principles through observation, interaction, and adaptive reasoning.
The project seeks to unify concepts from computational neuroscience, physics, and artificial intelligence to examine how machine systems may arrive at empirical laws similar to those discovered by human scientists.

\section*{Objectives}
\begin{itemize}
    \item Design artificial agents capable of self-guided experimentation and observation.
    \item Analyze the emergence of symbolic representations and laws from raw environmental data.
    \item Investigate the limits of synthetic intelligence in rediscovering established physical principles.
\end{itemize}

\section*{Repository Structure}
To be detailed as modules are implemented and documented.

\section*{License and Attribution}
This project is licensed under the \textbf{Apache License 2.0}.
All modifications, extensions, or derivative works must be properly documented to preserve the scientific integrity and intent of the original research.

\section*{Citation}
If you reference or build upon this work, please cite as:
\begin{quote}
Hicks, H. H. (2025). \textit{Project Lambda: Can Artificial Agents Rediscover the Laws of Physics?}
Tallahassee State College. \url{https://github.com/henryhicks/project-lambda}
\end{quote}

\section*{Contact}
For correspondence, inquiries, or collaboration requests, please contact:
\texttt{Hicks886@mymail.tsc.fl.edu}

% --- Add the epigraph as a footer on the last page ---
\afterpage{%
  \thispagestyle{fancy}%
  \fancyfoot[C]{\footnotesize\itshape
  "There comes a time when the mind takes a higher plane of knowledge but can never prove how it got there."\\
  --- Albert Einstein}
}

\end{document}
